\documentclass[12pt]{article}

\usepackage{fancyhdr}
\usepackage{geometry}
\usepackage{ucs}
\usepackage[utf8x]{inputenc}
\usepackage[T1]{fontenc}
\usepackage[ngerman]{babel}
\usepackage{amsmath,amssymb,amstext}
\usepackage{hyperref}
\usepackage{cancel}
\usepackage{dsfont}
\usepackage{physics}
\usepackage{lmodern}
\usepackage{enumerate}
\usepackage{enumitem}
\usepackage{graphicx}
\usepackage{listings, color}
\usepackage[labelfont=bf]{caption}
\usepackage{titling}

\lstset{basicstyle=\scriptsize} %Quellcode mit Umlauten und ganz klein
\lstset{literate=
  {Ö}{{\"O}}1
  {Ä}{{\"A}}1
  {Ü}{{\"U}}1
  {ß}{{\ss}}2
  {ü}{{\"u}}1
  {ä}{{\"a}}1
  {ö}{{\"o}}1
}


%Geometrie----------------------------------------------------------------------------------------------------------

\geometry{a4paper, top=25mm, left=15mm, right=15mm, bottom=25mm,headsep=10mm, footskip=10mm}
\pagestyle{fancy}
\setlength{\parindent}{0pt} %Zeileneinrückung

\fancyhf{} %Setzt voreingestellte Kopf-und Fußzeilen-Eigenschaften zurück

\lhead{\nouppercase{\leftmark}}
\chead{}
\rhead{\thepage}

\lfoot{}
\cfoot{}
\rfoot{}

\title{\vspace{0cm}{\Huge Fortgeschrittenen-Praktikum I:\\ \vspace{1cm} SQUID}}
\author{Saskia Bondza\\Simon Stephan}
\date{durchgeführt am 23.09.2016}

\pretitle{%
  \begin{center}
  \LARGE
  \includegraphics[width=6cm,]{figures/siegel}\\[\bigskipamount]
}
\posttitle{\end{center}}

%neue Commands----------------------------------------------------------------------------------------------------------
\newcommand{\nab}{\vec{\nabla}} %direkter Befehl mit Vektorpfeil
\newcommand{\gra}[3][0.7]{
	\begin{minipage}[h!]{\textwidth}
		\centering
		\includegraphics[width=#1\textwidth]{figures/#2.png}
		\captionof{figure}{#3}
	\end{minipage}
	\vskip 30 pt
}
\newcommand{\graTwo}[4][0.5]{
	\begin{minipage}[h!]{\textwidth}
		\centering
		\includegraphics[width=#1\textwidth]{figures/#2.png}
		\includegraphics[width=#1\textwidth]{figures/#3.png}
		\captionof{figure}{#4}
	\end{minipage}
	\vskip 30 pt
}
\newcommand{\code}[1]{\texttt{#1}}


%Titel,Inhalt----------------------------------------------------------------------------------------------------------

\begin{document}
\pagenumbering{gobble} %verstecke Seitenzahl
\maketitle
\newpage

\section*{Abstract}

In diesem Versuch soll die Anwendung von Supraleitern zum hochpräzisen  Messen von Magnetfeldern untersucht werden. Hierzu verwenden wir einen \textbf{S}uperconducting \textbf{QU}antum \textbf{I}nterference \textbf{D}evice) (SQUID), der im Wesentlichen aus einem Supraleiterring und einem Schwingkreis besteht. Von einigen Proben, darunter einer Leiterschleife mit fünf verschiedenen Widerständen, wird so das Dipolmoment und die Feldstärke gemessen.

\newpage

\thispagestyle{empty}
\tableofcontents
\newpage

%Schreiben----------------------------------------------------------------------------------------------------------
\pagenumbering{arabic} %verstecke Seitenzahl
\section{Einleitung}

Das SQUID (Superconducting QUantum Interference Device) ist ein Gerät, welches den Effekt der Supraleitung ausnutzt, um Magnetfelder zu messen. In diesem Versuch benutzen wir ein SQUID, um das Magnetfeld einer Leiterschleifen mit verschiedenen Widerständen und verschiedener sonstiger Gegenstände zu vermessen. Daraus bestimmen wir die Dipolmomente der Leiterschleife und der Gegenstände.


\newpage
\section{Theoretische Grundlagen}

\subsection{Supraleiter}\label{supraleiter}
Supraleiter sind auf keine Stoffklasse beschränkt und haben folgende charakteristische Eigenschaften:

\begin{itemize}
	\item Unterhalb einer kritischen Temperatur $T_c$ sinkt der elektrische Widerstand eines Supraleiters auf einen nicht messbaren Wert, ist also praktisch null
	\item Supraleiter sind perfekte Diamagneten - äußere Magnetfelder können durch die vom im Leiter induzierten Ströme und das hieraus entstehende innere Magnetfeld kompensiert werden REF 
	\item Die Elektronen eines Supraleiters gehen unterhalb der kritischen Temperatur paarweise Bindungen ein, diese sogenannten Cooper-Paare können als Bosonen behandelt werden wohingegen Elektronen Fermionen sind
	\item In der Bandstruktur befindet sich bei Supraleitern eine tempereaturabhängige Energielücke, Einbringen oder Dissoziation von Elektronen unter dieser Energie ist nicht möglich \label{bandluecke}
\end{itemize}

Oberhalb der kritischen Temperatur verhält sich der Widerstand eines Supraleiters wie ein Leiter:

\begin{align*}
\rho(T) = \rho_0 + k \cdot T^5
\end{align*}

\subsection{Magnetische Eigenschaften}
\gra{meissner-ochsenfeld}{Supraleiter im Magnetfeld (Meissner-Ochsenfeld-Effekt)\label{meissner-ochsenfeld}\textsuperscript{\cite{staat}}}
\subsubsection{Meissner-Ochsenfeld-Effekt}
Unterhalb der kritischen Temperatur $T_c$ wird ein angelegtes äußeres Magnetfeld durch ein inneres Magnetfeld aufgrund eines Abschirmstroms kompensiert. Diesen Effekt nennt man den Meissner-Ochsenfeld-Effekt\textsuperscript{\cite{anleitung}}. Innerhalb des Supraleiters gilt also für das Magnetfeld $\vec B=0$. Dies entspricht jedoch nicht einer perfekten Leitfähigkeit, da bei dieser das Magnetfeld erhalten bliebe. Dies ist bei Supraleitern des \textbf{Typ I} der Fall.
\subsubsection{Hochtemperatursupraleiter}
Supraleiter des \textbf{Typ II}, sogenannte Hochtemperatursupraleiter, können auch trotz eines inneren Magnetfelds Supraleitung aufweisen. In diesen Supraleitern bilden sich sogenannte Flussfäden aus, welche ein von Null verschiedenes Magnetfeld aufweisen und normalleitend sind. Der Rest des Supraleiters befindet sich jedoch im supraleitenden Zustand.

\subsection{Aufbrechen der Supraleitung}
\subsubsection{Kritische Temperatur}
Ein Supraleiter ist nur unterhalb einer kritischen Temperatur $T_c$ supraleitend. Oberhalb dieser Temperatur ist die Energie der Elektronen zu hoch und das Material ist nicht supraleitend.
\subsubsection{Kritische Feldstärke}
Wenn das äußere Magnetfeld die kritische Feldstärke $H_c$ überschreitet, reichen die induzierten Ströme nicht mehr aus, um das Magnetfeld zu kompensieren und die Supraleitung bricht zusammen. Bei Hochtemperatursupraleitern gibt es zwei kritische Feldstärken $H_{c_1}$ und  $H_{c_2}$. Bei Magnetfeldstärken $H>H_{c_2}$ bricht die Supraleitung auf, bei $H_{c_1}<H<H_{c_2}$ ist das Material supraleitend und bildet normalleitende Flussfäden aus. Bei $H<H_{c_1}$ ist das Material vollständig supraleitend.
\subsubsection{Große Ströme}
Ströme innerhalb des Supraleiters induzieren ein Magnetfeld. Ab einer gewissen Stromstärke überschreitet das induzierte Magnetfeld die kritische Feldstärke und führt dadurch zu einem Aufbrechen der Supraleitung.
\subsubsection{Anregung der Elektronen}
Die Elektronen können auch durch ein äußeres elektromagnetisches Wechselfeld angeregt werden. Mit einer Frequenz des Wechselfelds von $\omega\approx\frac{\Delta E}{\hbar}$ werden die Elektronen über die Bandlücke (siehe \ref{bandluecke}) hinaus angeregt und die Fähigkeit zur Supraleitung geht verloren.

\subsection{London-Gleichungen}

\subsection{BCS-Theorie}

\subsection{Flussquantisierung}

\subsection{SQUID}



\newpage
\section{Versuchsaufbau und -Durchführung}




\newpage
\section{Auswertung}


\newpage
\section{Zusammenfassung und Diskussion}


\newpage
\section{Anhang}

%\subsection{Grafiken}


%\subsection{Tabellen}

%\subsubsection{$\alpha$-Plateau Samarium}
%\lstinputlisting[language=MATLAB]{Rohdaten/alphaPlateau_Sm.txt}


%\newpage
%\subsection{Quellcode (MATLAB)}
%\lstinputlisting[language=MATLAB]{Rohdaten/alpha.m}

%\newpage
\subsection{Laborheft}
%\begin{minipage}{\textwidth}
%\centering
%\includegraphics[width=0.9\textwidth]{figures/IMG_20151002_141014.jpg}
%\end{minipage}

\newpage
\listoffigures

%Literatur----------------------------------------------------------------------------------------------------------

%\cite{les}
\newpage
\thispagestyle{empty}
\begin{thebibliography}{9}

%\bibitem{staat}
%  Tobijas Kotyk,
%  \emph{Versuche zur Radioaktivität im Physikalischen Fortgeschrittenen Praktikum an der Albert-Ludwigs-Universität Freiburg},
%  Albert-Ludwigs-Universität, Freiburg,
%  2005
  

  
%\bibitem{molmasse}
%  \emph{http://www.convertunits.com/molarmass/<ELEMENTNAME AUF ENGLISCH>}, Stand 28.09.2015
  

\bibitem{anleitung}
M. Köhli,
\emph{Versuchsanleitung Fortgeschrittenen Praktikum: SQUID},
Albert-Ludwigs-Universität Freiburg,
2011

\bibitem{staat}
Volker Bange,
\emph{Einrichtung des Versuches "SQUID"},
Albert-Ludwigs-Universität Freiburg,
2000
\end{thebibliography}

\end{document}