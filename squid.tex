\documentclass[12pt]{article}

\usepackage{fancyhdr}
\usepackage{geometry}
\usepackage{ucs}
\usepackage[utf8x]{inputenc}
\usepackage[T1]{fontenc}
\usepackage[ngerman]{babel}
\usepackage{amsmath,amssymb,amstext}
\usepackage{hyperref}
\usepackage{cancel}
\usepackage{dsfont}
\usepackage{physics}
\usepackage{lmodern}
\usepackage{enumerate}
\usepackage{enumitem}
\usepackage{graphicx}
\usepackage{listings, color}
\usepackage[labelfont=bf]{caption}
\usepackage{titling}

\lstset{basicstyle=\scriptsize} %Quellcode mit Umlauten und ganz klein
\lstset{literate=
  {Ö}{{\"O}}1
  {Ä}{{\"A}}1
  {Ü}{{\"U}}1
  {ß}{{\ss}}2
  {ü}{{\"u}}1
  {ä}{{\"a}}1
  {ö}{{\"o}}1
}


%Geometrie----------------------------------------------------------------------------------------------------------

\geometry{a4paper, top=25mm, left=15mm, right=15mm, bottom=25mm,headsep=10mm, footskip=10mm}
\pagestyle{fancy}
\setlength{\parindent}{0pt} %Zeileneinrückung

\fancyhf{} %Setzt voreingestellte Kopf-und Fußzeilen-Eigenschaften zurück

\lhead{\nouppercase{\leftmark}}
\chead{}
\rhead{\thepage}

\lfoot{}
\cfoot{}
\rfoot{}

\title{\vspace{0cm}{\Huge Fortgeschrittenen-Praktikum I:\\ \vspace{1cm} SQUID}}
\author{Saskia Bondza\\Simon Stephan}
\date{durchgeführt am 23.09.2016}

\pretitle{%
  \begin{center}
  \LARGE
  \includegraphics[width=6cm,]{figures/siegel}\\[\bigskipamount]
}
\posttitle{\end{center}}

%neue Commands----------------------------------------------------------------------------------------------------------
\newcommand{\nab}{\vec{\nabla}} %direkter Befehl mit Vektorpfeil
\newcommand{\gra}[3][0.7]{
	\begin{minipage}[h!]{\textwidth}
		\centering
		\includegraphics[width=#1\textwidth]{figures/#2.png}
		\captionof{figure}{#3}
	\end{minipage}
	\vskip 30 pt
}
\newcommand{\graTwo}[4][0.49]{
	\begin{minipage}[h!]{\textwidth}
		\centering
		\includegraphics[width=#1\textwidth]{figures/#2.png}
		\includegraphics[width=#1\textwidth]{figures/#3.png}
		\captionof{figure}{#4}
	\end{minipage}
	\vskip 30 pt
}
\newcommand{\graTwoB}[5]{
	\begin{minipage}[h!]{\textwidth}
		\centering
		\includegraphics[width=#1\textwidth]{figures/#3.png}
		\includegraphics[width=#2\textwidth]{figures/#4.png}
		\captionof{figure}{#5}
	\end{minipage}
	\vskip 30 pt
}
\newcommand{\del}[2][]{\frac{\partial #1}{\partial #2}}
\newcommand{\code}[1]{\texttt{#1}}


%Titel,Inhalt----------------------------------------------------------------------------------------------------------

\begin{document}
\pagenumbering{gobble} %verstecke Seitenzahl
\maketitle
\newpage

\section*{Abstract}

In diesem Versuch soll die Anwendung von Supraleitern zum hochpräzisen  Messen von Magnetfeldern untersucht werden. Hierzu verwenden wir einen \textbf{S}uperconducting \textbf{QU}antum \textbf{I}nterference \textbf{D}evice (SQUID), der im Wesentlichen aus einem Supraleiterring und einem Schwingkreis besteht. Von einigen Proben, darunter einer Leiterschleife mit fünf verschiedenen Widerständen, wird so das Dipolmoment und die Feldstärke gemessen.

\newpage

\thispagestyle{empty}
\tableofcontents
\newpage

%Schreiben----------------------------------------------------------------------------------------------------------
\pagenumbering{arabic} %verstecke Seitenzahl
\section{Einleitung}

Das SQUID (Superconducting QUantum Interference Device) ist ein Gerät, welches den Effekt der Supraleitung ausnutzt, um Magnetfelder zu messen. In diesem Versuch benutzen wir ein SQUID, um das Magnetfeld einer Leiterschleifen mit verschiedenen Widerständen und verschiedener sonstiger Gegenstände zu vermessen. Daraus bestimmen wir die Dipolmomente der Leiterschleife und der Gegenstände. Außerdem betrachten wir das Magnetfeld in Abhängigkeit des Drehwinkels der Proben.


\newpage
\section{Theoretische Grundlagen}


\subsection{Supraleiter}\label{supraleiter}
Supraleiter sind auf keine Stoffklasse beschränkt und haben folgende charakteristische Eigenschaften:

\begin{itemize}
	\item Unterhalb einer kritischen Temperatur $T_c$ sinkt der elektrische Widerstand eines Supraleiters auf einen nicht messbaren Wert, ist also praktisch null
	\item Supraleiter sind perfekte Diamagneten - äußere Magnetfelder können durch die vom im Leiter induzierten Ströme und das hieraus entstehende innere Magnetfeld kompensiert werden REF 
	\item Die Elektronen eines Supraleiters gehen unterhalb der kritischen Temperatur paarweise Bindungen ein, diese sogenannten Cooper-Paare können als Bosonen behandelt werden wohingegen Elektronen Fermionen sind
	\item In der Bandstruktur befindet sich bei Supraleitern eine temperaturabhängige Energielücke, Einbringen oder Dissoziation von Elektronen unter dieser Energie ist nicht möglich \label{bandluecke}
\end{itemize}

Oberhalb der kritischen Temperatur verhält sich der Widerstand eines Supraleiters wie ein Leiter (siehe Abbildung \ref{tempwid}):

\begin{align*}
\rho(T) = \rho_0 + k \cdot T^5
\end{align*}

\gra{temperatur-widerstand}{Temperaturverhalten des elektrischen Widerstands eines Supraleiters\textsuperscript{\cite{chemistry}}\label{tempwid}}

\subsection{Magnetische Eigenschaften}
\gra{meissner-ochsenfeld}{Supraleiter im Magnetfeld (Meissner-Ochsenfeld-Effekt)\label{meissner-ochsenfeld}\textsuperscript{\cite{staat}}}
\subsubsection{Meissner-Ochsenfeld-Effekt}
Unterhalb der kritischen Temperatur $T_c$ wird ein angelegtes äußeres Magnetfeld durch ein inneres Magnetfeld aufgrund eines Abschirmstroms kompensiert. Diesen Effekt nennt man den Meissner-Ochsenfeld-Effekt\textsuperscript{\cite{anleitung}}. Innerhalb des Supraleiters gilt also für das Magnetfeld $\vec B=0$. Dies entspricht jedoch nicht einer perfekten Leitfähigkeit, da bei dieser das Magnetfeld erhalten bliebe. Bei Supraleitern des \textbf{Typ I} wird das Magnetfeld wie hier beschrieben komplett kompensiert.
\subsubsection{Hochtemperatursupraleiter}
Supraleiter des \textbf{Typ II}, sogenannte Hochtemperatursupraleiter, können auch trotz eines inneren Magnetfelds Supraleitung aufweisen. In diesen Supraleitern bilden sich sogenannte Flussfäden aus, welche ein von Null verschiedenes Magnetfeld aufweisen und normalleitend sind. Der Rest des Supraleiters befindet sich im supraleitenden Zustand und weist ein Magnetfeld $\vec B=0$ auf.

\subsection{Aufbrechen der Supraleitung}
\subsubsection{Kritische Temperatur}
Ein Supraleiter ist nur unterhalb einer kritischen Temperatur $T_c$ supraleitend. Oberhalb dieser Temperatur ist die Energie der Elektronen zu hoch und das Material ist nicht supraleitend.
\subsubsection{Kritische Feldstärke}
Wenn das äußere Magnetfeld die kritische Feldstärke $H_c$ überschreitet, reichen die induzierten Ströme nicht mehr aus, um das Magnetfeld zu kompensieren und die Supraleitung bricht zusammen. Bei Hochtemperatursupraleitern gibt es zwei kritische Feldstärken $H_{c_1}$ und  $H_{c_2}$. Bei Magnetfeldstärken $H>H_{c_2}$ bricht die Supraleitung auf, bei $H_{c_1}<H<H_{c_2}$ ist das Material supraleitend und bildet normalleitende Flussfäden aus. Bei $H<H_{c_1}$ ist das Material vollständig supraleitend.
\subsubsection{Große Ströme}
Ströme innerhalb des Supraleiters induzieren ein Magnetfeld. Ab einer gewissen Stromstärke überschreitet das induzierte Magnetfeld die kritische Feldstärke und führt dadurch zu einem Aufbrechen der Supraleitung.
\subsubsection{Anregung der Elektronen}
Die Elektronen können auch durch ein äußeres elektromagnetisches Wechselfeld angeregt werden. Mit einer Frequenz des Wechselfelds von $\omega\approx\frac{\Delta E}{\hbar}$ werden die Elektronen über die Bandlücke (siehe \ref{bandluecke}) hinaus angeregt und die Fähigkeit zur Supraleitung geht verloren.

\subsection{London-Gleichungen\textsuperscript{\cite{anleitung}}}
\label{London}
Für das Eindringen des Magnetfelds in einen Supraleiter gilt die in \cite{anleitung} hergeleitete London-Gleichung:
\begin{align}
	\nabla\times\vec j=-\frac{n_ee^2}{m}\vec B
\end{align}
Daraus folgen folgende Gleichungen:
\begin{align*}
	\nabla^2\vec B&=\frac{\mu_0n_ee^2}{m}\vec{B}\\
	\nabla^2\vec j&=\frac{\mu_0n_ee^2}{m}\vec{j}
\end{align*}
Die Lösungen dieser Gleichungen sind abfallende Exponentialfunktionen mit einer Reichweite von $\Lambda=\sqrt{\frac m{\mu_0n_ee^2}}$. Aus den Lösungen folgt, dass ein Magnetfeld ein gewisses Stück in den Supraleiter eindringen kann und in dieser Schicht senkrecht zum Magnetfeld eine Stromdichte $\vec j$ fließt, welche das Magnetfeld abschirmt.

\subsection{BCS-Theorie}
\subsubsection{Überabschirmung der Coulomb-Wechselwirkung}\label{ueberabschirmung}
Elektronen in einem Kristallgitter wirken eine Anziehungskraft auf die positiv geladenen Atomrümpfe aus. Bei Bewegung der Elektronen werden die Atomrümpfe nun ausgelenkt. Durch ihre größere Masse sind sie langsamer als die Elektronen und brauchen daher länger, um zu ihrer Ursprungsposition zurückzukehren. Deshalb ziehen die Elektronen nun eine Polarisationsspur hinter sich her. Die positiv geladenen Atomrümpfe wechselwirken nun anziehend mit anderen Elektronen, welches eine effektive anziehende Wechselwirkung zwischen zwei Elektronen zur Folge hat. Die Wechselwirkung zweier Elektronen mit den Wellenvektoren $\vec k$ und $\vec k'$ und der Energiedifferenz $\Delta E=\hbar\omega$ lässt sich durch folgenden Ausdruck darstellen: \textsuperscript{\cite{anleitung}}
\begin{align}
	V_{\vec k,\vec k'}&=4\pi\cdot\frac{e^2}{q^2-k_0^2}\cdot\frac{\omega^2}{\omega^2-\omega_D^2} & \text{mit }k_0^2&=4\pi e^2\frac{\delta n_0}{\delta \mu}\label{cooper-potential}
\end{align}
\subsubsection{Cooper-Paar}
Durch diese Überabschirmung der Coulomb-Wechselwirkung (siehe \ref{ueberabschirmung}) bilden sich nun Paare von Elektronen aus, welche durch die effektive Anziehung schwach gebunden sind. Das Potential aus Gleichung (\ref{cooper-potential}) ist das Potential der effektiven Anziehung. Wenn nun $\Delta E\approx\hbar\omega_D$, kann sich eine vergleichsweise starke Anziehung ergeben.\\

Elektronen sind Fermionen, sie haben den Spin $S=\frac12$. Fermionen unterliegen der Fermi-Statistik, das heißt bis zur Energie $E_F$ sind fast alle Zustände besetzt, danach sinkt die Besetzungsdichte gegen Null.
Ein Cooper-Paar als Verkettung zweier $S_\frac12$ - Zustände kann nun als Boson mit ganzzahligem Spin betrachtet werden. Dadurch entsteht in dem Bereich $E_F-\hbar\omega_D<E<E_F+\hbar\omega_D$ eine Energie-Bandlücke.\textsuperscript{\cite{anleitung}}
\subsubsection{Wechselwirkung zwischen Cooper-Paaren}
Die Wellenfunktion eines Cooper-Paares lässt sich darstellen als: \textsuperscript{\cite{anleitung}}
\begin{align}
	\Phi(\vec{r_1}-\vec{r_2})&=\frac1{V_{\mathrm{ges}}}\sum_{\vec k}\phi_{\vec k}e^{i\vec k(\vec{r_1}-\vec{r_2})}
\end{align}
Nun wird folgende Wechselwirkung zwischen zwei Cooper-Paaren definiert:
\begin{align}
	\bra{\vec{k_1}\vec{k_2}}V\ket{\vec{k_3}\vec{k_4}}&=\begin{cases}-\frac{V_0}{L^3}&\vec{k_1}+\vec{k_2}=\vec{k_3}+\vec{k_4} \text{ und }|E(\vec{k_i})-E_F|<\hbar\omega\\0&\text{sonst}\end{cases}
\end{align}

\subsection{Flussquantisierung}
Ebenfalls zur Beschreibung von Supraleitern lässt sich die Ginzburg-Landau-Theorie\textsuperscript{\cite{anleitung}} nutzen. In dieser Theorie betracht man den Zwei-Teilchen-Zustand eines Cooper-Paares als ein einziges Gesamtsystem. Nun lässt sich ein Ausdruck für den Zwei-Teilchen-Strom bestimmen\textsuperscript{\cite{anleitung}}:
\begin{align}
	\vec j&=\frac{i\hbar}{2m}\left(\psi\nabla\psi^* -\psi*\nabla\psi\right)\\
	&=\frac1m\Re{\psi^*p\psi}&\text{mit }&p=-i\hbar\nabla-\frac qc\vec A\\
	&=-\frac{e}{2m}\left(\psi*\left(\left(-i\hbar\nabla-\frac qc\vec A\right)\psi\right)+c.c.\right)\\
	&=-\left(\frac{2e^2}{mc}\vec A+\frac{e\hbar}{m}\nabla\theta\right)|\psi|^2&\text{mit }&\psi=|\psi|e^{i\theta}\text{ (}|\psi|\text{ konstant)}\label{ginzburg-landau}
\end{align}
Bei einem Supraleiter in Form eines Rings lässt sich der Strom als geschlossenes Wegintegral $$\oint_C\vec j\mathrm d\vec l=0$$ darstellen, welches Null ergeben muss, da ansonsten ein Potentialgradient oder eine Stromänderung vorhanden wäre. Nach dem Stokes'schen Satz ist der magnetische Fluss $\Phi_B$ gegeben durch das Wegintegral über das Vektorpotential $\vec{A}$. Außerdem ist die Phase der Wellenfunktion eindeutig festgelegt und darf sich nur um ganzzahlige Vielfache von $2\pi$ ändern. Also gilt mit Gleichung (\ref{ginzburg-landau}):
\begin{align*}
	\oint_C\vec A\mathrm d\vec l&=\Phi_B\\
	\oint_C\nabla\theta\mathrm d\vec l&=\Delta\theta=2\pi n
\end{align*}
Daraus folgt nun die Quantisierung für den magnetischen Fluss:
\begin{align*}
	|\Phi_B|&=n\frac h{2e}\\
	&=n\Phi_0
\end{align*}
$\Phi_0$ ist das sogenannte Flussquant, oder auch Fluxoid\textsuperscript{\cite{anleitung}}.
\subsection{Josephson-Effekt}
Wenn man zwischen zwei Supraleitern einen Isolator einbringt, fließt nun aufgrund der im Isolator exponentiell abfallenden Aufenthaltswahrscheinlichkeiten der Cooper-Paare und der einzelnen Elektronen im Isolator ein Tunnelstrom. Um nun die Gesamtwellenfunktion beschreiben zu können, muss die Wellenfunktion $\Psi_1$ des einen Supraleiters stetig in die Wellenfunktion $\Psi_2$ des anderen Supraleiters übergehen. 

\subsection{SQUID}
Das SQUID (Superconducting QUantum Interference Device) ist ein Gerät zur Messung von kleinen Magnetfeldänderungen, welches die Flussquantifizierung und andere Effekte der Supraleitung nutzt. SQUIDs existieren in verschiedenen Bauweisen. In diesem Versuch nutzen wir ein RF-SQUID (Radio Frequency SQUID).

\gra{squid}{Aufbau eines RF-SQUIDs}

Das SQUID besteht aus einem supraleitendem Ring mit Josephson-Kontakt und einem hochfrequenten elektromagnetischen Schwingkreis. Der Ring wird mit Stickstoff gekühlt, um unter die kritische Temperatur zu kommen.\\

Von dem Schwingkreis wird ein magnetischer Fluss erzeugt. Im SQUID bilden sich nun Kreisströme aus, die das äußere Feld ausgleichen. Diese Kreisströme entsprechen immer genau einem ganzzahligen Vielfachen von $\Phi_0$. Wenn der benötigte Ausgleichsstrom nun nicht exakt einem ganzzahligen Vielfachen von $\Phi_0$ entspricht, bildet sich ein Abschirmstrom auf der Oberfläche (siehe \ref{London}), welcher einen vollständigen Ausgleich des äußeren Felds bewirkt.\\

Wenn der Abschirmstrom nun zu hoch wird, bricht die Supraleitung zusammen. Der Suprastrom im Inneren des Supraleiterrings kann nun von $n\Phi_0$ auf $(n+1)\Phi_0$ ansteigen, wodurch der Abschirmstrom abnimmt und die Supraleitung wieder einsetzt. Der Schwingkreis verliert durch diesen Vorgang Energie, was über das Oszilloskop sichtbar wird.

Mit dem Feld-Fluss-Koeffizienten $F=9.3\,\mathrm{\frac{nT}{\Phi_0}}$ und dem einstellbaren Wert des Feedback-Resistors $s_i$ mit $[s_i]=\mathrm{\frac{V}{\Phi_0}}$ lässt sich nun aus der gemessenen Spannungsdifferenz $\Delta V$ das Magnetfeld berechnen:
\begin{align}
	B_z&=F\frac{\Delta V}{s_i}
\end{align}

\subsection{Magnetfeld einer Leiterschleife}
Eine stromdurchflossene Leiterschleife induziert ein Magnetfeld. Die magnetische Flussdichte des Magnetfelds im Abstand $z$ zur Leiterschleife beträgt:
\begin{align*}
	B_z&=\frac{\mu_0p}{2\pi z^3}
\end{align*}
Mit dem Dipolmoment $p=AI=A\frac V{R_i}$, wobei $A$ die eingeschlossene Fläche der Leiterschleife ist, ergibt sich:
\begin{align}
B_z&=\frac{\mu_0AV}{2\pi z^3R_i}
\end{align}


\newpage
\section{Versuchsaufbau und -Durchführung}

\gra[0.8]{aufbau}{Schematische Zeichnung des Versuchsaufbaus\label{aufbau}}

In Abbildung \ref{aufbau} ist der Versuchsaufbau zu sehen. Das Kernstück dieses Aufbaus ist das SQUID in einem isolierenden Behältnis, welches zur Kühlung des SQUID mit flüssigem Stickstoff gefüllt ist. Mit einem Stab werden die Proben in das Gehäuse nahe an das SQUID geführt. Der Stab ist an einem Motor befestigt. Das SQUID ist über eine elektronische Auslese- und Steuerungseinheit mit einem Computer und einem Oszilloskop verbunden.\\

Zunächst wird das SQUID kalibriert, um möglichst gute Ergebnisse zu erzielen. Dazu wird über einen Funktionsgenerator innerhalb der SQUID-Elektronik ein Dreiecksignal erzeugt, welches eine Sinuskurve der Spannung bewirkt. Nun werden die Parameter VCA und VCO des SQUIDs mithilfe des Computerprogramms so angepasst, dass die Amplitude der Spannung möglichst groß ist.\\

Anschließend wird das Magnetfeld einer Leiterschleife mit verschiedenen Widerständen vermessen. Dazu wird ein Stab mit der Leiterschleife an den Motor angeschlossen und nun mit konstanter Geschwindigkeit um die eigene Achse gedreht. So entsteht eine Änderung des Magnetfelds, aus welcher wir das Magnetfeld bestimmen können.\\

Diese Messung wird danach mit mehreren Gegenständen durchgeführt. Wir benutzen hier einen Kronkorken, einen Ohrring, einen Spitzer, ein Geldstück, einen Magnetspan und einen Stein. Für den Kronkorken führen wir die Messung mehrmals bei verschiedenen Abständen durch. Den Ohrring und den Spitzer befestigten wir wie in Abbildung \ref{ohrspitzer} zu sehen.

\vskip10pt
\graTwo[0.35]{ohrring}{spitzer}{\label{ohrspitzer}Befestigung von Ohrring und Spitzer}
\newpage
\section{Auswertung}
\subsection{Leiterschleife}
\graTwoB{0.65}{0.34}{R1_Sinusfit}{R1_polar-plot}{Widerstand R1}
\graTwoB{0.65}{0.34}{R2_Sinusfit}{R2_polar-plot}{Widerstand R2}
\graTwoB{0.65}{0.34}{R3_Sinusfit}{R3_polar-plot}{Widerstand R3}
\graTwoB{0.65}{0.34}{R4_Sinusfit}{R4_polar-plot}{Widerstand R4}
\graTwoB{0.65}{0.34}{R5_Sinusfit}{R5_polar-plot}{Widerstand R5}
\subsection{Gegenstände}
\subsubsection{Geldstück}
\graTwoB{0.65}{0.34}{Geld_Sinusfit}{Geld_polar-plot}{Geldstück}
Die gefittete Sinus-Funktion für das Geldstück lautet wie folgt (siehe Gleichung (\ref{U})):
\begin{align*}
	U(t)=(-3.3596\pm0.0014)\mathrm V-(0.3501\pm0.0019)\mathrm V\cdot \sin((0.8717 \pm 0.0011)\mathrm{s^{-1}}\cdot t + (6.967 \pm 0.012))
\end{align*}
Daraus folgt für das Magnetfeld und das Dipolmoment des Geldstücks mit den Gleichung (\ref{B}) und (\ref{p}):
\begin{align*}
	B_z&=(858\pm5)\,\mathrm{pT}\\
	p&=(2.21\pm0.10)\,\mathrm{mm^2A}
\end{align*}
\newpage
\subsubsection{Magnetspan}
\graTwoB{0.65}{0.34}{Magnetspan_Sinusfit}{Magnetspan_polar-plot}{Magnetspan}
Die gefittete Sinus-Funktion für den Magnetspan lautet wie folgt (siehe Gleichung (\ref{U})):
\begin{align*}
	U(t)=(1.8493\pm0.0017)\mathrm V+(1.004\pm0.002)\mathrm V\cdot \sin((0.8671 \pm 0.0004)\mathrm{s^{-1}}\cdot t + (8.729 \pm 0.005))
\end{align*}
Daraus folgt für das Magnetfeld und das Dipolmoment des Magnetspans mit den Gleichung (\ref{B}) und (\ref{p}):
\begin{align*}
	B_z&=(2458\pm6)\,\mathrm{pT}\\
	p&=(3.39\pm0.18)\,\mathrm{mm^2A}
\end{align*}
\newpage
\subsubsection{Spitzer}
\graTwoB{0.65}{0.34}{Spitzer_Sinusfit}{Spitzer_polar-plot}{Spitzer}
Die gefittete Sinus-Funktion für den Spitzer lautet wie folgt (siehe Gleichung (\ref{U})):
\begin{align*}
	U(t)=(-0.6547\pm0.0005)\mathrm V+(0.4498\pm0.0007)\mathrm V\cdot \sin((0.8710 \pm 0.0002)\mathrm{s^{-1}}\cdot t + (4.402 \pm 0.003))
\end{align*}
Daraus folgt für das Magnetfeld und das Dipolmoment des Spitzers mit den Gleichung (\ref{B}) und (\ref{p}):
\begin{align*}
	B_z&=(1100.8\pm1.6)\,\mathrm{pT}\\
	p&=(2.84\pm0.12)\,\mathrm{mm^2A}
\end{align*}
\newpage
\subsubsection{Stein}
\graTwoB{0.65}{0.34}{Stein_Sinusfit}{Stein_polar-plot}{Stein}
Die gefittete Sinus-Funktion für den Stein lautet wie folgt (siehe Gleichung (\ref{U})):
\begin{align*}
	U(t)=(-2.3786\pm0.0008)\mathrm V+(0.1211\pm0.0011)\mathrm V\cdot \sin((0.8466 \pm 0.0016)\mathrm{s^{-1}}\cdot t + (6.330 \pm 0.018))
\end{align*}
Daraus folgt für das Magnetfeld und das Dipolmoment des Steins mit den Gleichung (\ref{B}) und (\ref{p}):
\begin{align*}
	B_z&=(296\pm3)\,\mathrm{pT}\\
	p&=(0.77\pm0.03)\,\mathrm{mm^2A}
\end{align*}
\newpage
\subsubsection{Ohrring}
\graTwoB{0.65}{0.34}{Ohrring_Sinusfit}{Ohr_polar-plot}{Ohrring}
Die gefittete Sinus-Funktion für den Ohrring lautet wie folgt (siehe Gleichung (\ref{U})):
\begin{align*}
	U(t)=(-7.041\pm0.006)\mathrm V+(1.547\pm0.008)\mathrm V\cdot \sin((0.8756 \pm 0.0008)\mathrm{s^{-1}}\cdot t + (5.392 \pm 0.010))
\end{align*}
Daraus folgt für das Magnetfeld und das Dipolmoment des Ohrrings mit den Gleichung (\ref{B}) und (\ref{p}):
\begin{align*}
	B_z&=(3786\pm19)\,\mathrm{pT}\\
	p&=(9.8\pm0.4)\,\mathrm{mm^2A}
\end{align*}

\newpage
\subsection{Kronkorken}
\subsubsection{Kronkorken bei $z=33.5\,\mathrm{cm}$}
\graTwoB{0.65}{0.34}{KK_33-5_Sinusfit}{KK_33-5_polar-plot}{Kronkorken bei $z=33.5\,\mathrm{cm}$}
Die gefittete Sinus-Funktion für den Kronkorken bei $z=33.5\,\mathrm{cm}$ lautet wie folgt (siehe Gleichung (\ref{U})):
\begin{align*}
U(t)=(-4.09\pm0.04)\mathrm V+(10.68\pm0.06)\mathrm V\cdot \sin((0.87196 \pm 0.00010)\mathrm{s^{-1}}\cdot t + (9.394 \pm 0.011))
\end{align*}
Daraus folgt für das Magnetfeld und das Dipolmoment des Kronkorkens bei $z=33.5\,\mathrm{cm}$ mit den Gleichung (\ref{B}) und (\ref{p}):
\begin{align*}
B_z&=(26.13\pm0.15)\,\mathrm{nT}\\
p&=(42\pm2)\,\mathrm{mm^2A}
\end{align*}
\newpage
\subsubsection{Kronkorken bei $z=34.5\,\mathrm{cm}$}
\graTwoB{0.65}{0.34}{KK_34-5_Sinusfit}{KK_34-5_polar-plot}{Kronkorken bei $z=34.5\,\mathrm{cm}$}
Die gefittete Sinus-Funktion für den Kronkorken bei $z=33.5\,\mathrm{cm}$ lautet wie folgt (siehe Gleichung (\ref{U})):
\begin{align*}
U(t)=(-5.203\pm0.017)\mathrm V-(8.91\pm0.02)\mathrm V\cdot \sin((0.8675 \pm 0.0005)\mathrm{s^{-1}}\cdot t + (7.642 \pm 0.006))
\end{align*}
Daraus folgt für das Magnetfeld und das Dipolmoment des Kronkorkens bei $z=34.5\,\mathrm{cm}$ mit den Gleichung (\ref{B}) und (\ref{p}):
\begin{align*}
B_z&=(21.81\pm0.06)\,\mathrm{nT}\\
p&=(48\pm2)\,\mathrm{mm^2A}
\end{align*}
\newpage
\subsubsection{Kronkorken bei $z=35.0\,\mathrm{cm}$}
\graTwoB{0.65}{0.34}{Kronkorken_Sinusfit}{KK_35-0_polar-plot}{Kronkorken bei $z=35.0\,\mathrm{cm}$}
Die gefittete Sinus-Funktion für den Kronkorken bei $z=33.5\,\mathrm{cm}$ lautet wie folgt (siehe Gleichung (\ref{U})):
\begin{align*}
U(t)=(1.158\pm0.009)\mathrm V+(3.615\pm0.012)\mathrm V\cdot \sin((0.8728 \pm 0.0007)\mathrm{s^{-1}}\cdot t + (7.013 \pm 0.008))
\end{align*}
Daraus folgt für das Magnetfeld und das Dipolmoment des Kronkorkens bei $z=35.0\,\mathrm{cm}$ mit den Gleichung (\ref{B}) und (\ref{p}):
\begin{align*}
B_z&=(8.85\pm0.03)\,\mathrm{nT}\\
p&=(22.8\pm1.0)\,\mathrm{mm^2A}
\end{align*}
\newpage
\subsubsection{Kronkorken bei $z=35.5\,\mathrm{cm}$}
\graTwoB{0.65}{0.34}{KK_35-5_Sinusfit}{KK_35-5_polar-plot}{Kronkorken bei $z=35.5\,\mathrm{cm}$}
Die gefittete Sinus-Funktion für den Kronkorken bei $z=33.5\,\mathrm{cm}$ lautet wie folgt (siehe Gleichung (\ref{U})):
\begin{align*}
U(t)=(-8.757\pm0.009)\mathrm V+(2.441\pm0.013)\mathrm V\cdot \sin((0.8659 \pm 0.0009)\mathrm{s^{-1}}\cdot t + (8.837 \pm 0.010))
\end{align*}
Daraus folgt für das Magnetfeld und das Dipolmoment des Kronkorkens bei $z=35.5\,\mathrm{cm}$ mit den Gleichung (\ref{B}) und (\ref{p}):
\begin{align*}
B_z&=(5.97\pm0.03)\,\mathrm{nT}\\
p&=(17.9\pm0.7)\,\mathrm{mm^2A}
\end{align*}
\newpage
\subsubsection{Kronkorken bei $z=36.0\,\mathrm{cm}$}
\graTwoB{0.65}{0.34}{KK_36-0_Sinusfit}{KK_36-0_polar-plot}{Kronkorken bei $z=36.0\,\mathrm{cm}$}
Die gefittete Sinus-Funktion für den Kronkorken bei $z=33.5\,\mathrm{cm}$ lautet wie folgt (siehe Gleichung (\ref{U})):
\begin{align*}
U(t)=(-4.595\pm0.005)\mathrm V-(1.483\pm0.007)\mathrm V\cdot \sin((0.8649 \pm 0.0009)\mathrm{s^{-1}}\cdot t + (7.480 \pm 0.010))
\end{align*}
Daraus folgt für das Magnetfeld und das Dipolmoment des Kronkorkens bei $z=36.0\,\mathrm{cm}$ mit den Gleichung (\ref{B}) und (\ref{p}):
\begin{align*}
B_z&=(3.630\pm0.017)\,\mathrm{nT}\\
p&=(12.6\pm0.5)\,\mathrm{mm^2A}
\end{align*}
\newpage
\subsubsection{Kronkorken bei $z=36.5\,\mathrm{cm}$}
\graTwoB{0.65}{0.34}{KK_36-5_Sinusfit}{KK_36-5_polar-plot}{Kronkorken bei $z=36.5\,\mathrm{cm}$}
Die gefittete Sinus-Funktion für den Kronkorken bei $z=33.5\,\mathrm{cm}$ lautet wie folgt (siehe Gleichung (\ref{U})):
\begin{align*}
U(t)=(-7.754\pm0.004)\mathrm V-(0.693\pm0.006)\mathrm V\cdot \sin((0.8595 \pm 0.0015)\mathrm{s^{-1}}\cdot t + (7.710 \pm 0.018))
\end{align*}
Daraus folgt für das Magnetfeld und das Dipolmoment des Kronkorkens bei $z=36.5\,\mathrm{cm}$ mit den Gleichung (\ref{B}) und (\ref{p}):
\begin{align*}
B_z&=(1.696\pm0.015)\,\mathrm{nT}\\
p&=(6.7\pm0.3)\,\mathrm{mm^2A}
\end{align*}

\newpage
\section{Zusammenfassung und Diskussion}


\newpage
\section{Anhang}

%\subsection{Grafiken}


%\subsection{Tabellen}

%\subsubsection{$\alpha$-Plateau Samarium}
%\lstinputlisting[language=MATLAB]{Rohdaten/alphaPlateau_Sm.txt}


%\newpage
%\subsection{Quellcode (MATLAB)}
%\lstinputlisting[language=MATLAB]{Rohdaten/alpha.m}

%\newpage
\subsection{Laborheft}
%\begin{minipage}{\textwidth}
%\centering
%\includegraphics[width=0.9\textwidth]{figures/IMG_20151002_141014.jpg}
%\end{minipage}

\newpage
\listoffigures

%Literatur----------------------------------------------------------------------------------------------------------

%\cite{les}
\newpage
\thispagestyle{empty}
\begin{thebibliography}{9}

%\bibitem{staat}
%  Tobijas Kotyk,
%  \emph{Versuche zur Radioaktivität im Physikalischen Fortgeschrittenen Praktikum an der Albert-Ludwigs-Universität Freiburg},
%  Albert-Ludwigs-Universität, Freiburg,
%  2005
  

  
%\bibitem{molmasse}
%  \emph{http://www.convertunits.com/molarmass/<ELEMENTNAME AUF ENGLISCH>}, Stand 28.09.2015
  

\bibitem{anleitung}
M. Köhli,
\emph{Versuchsanleitung Fortgeschrittenen Praktikum: SQUID},
Albert-Ludwigs-Universität Freiburg,
2011

\bibitem{staat}
Volker Bange,
\emph{Einrichtung des Versuches "SQUID"},
Albert-Ludwigs-Universität Freiburg,
2000

\bibitem{chemistry}
Bruce A. Averill, Patricia Eldredge,
\emph{General Chemistry: Principles, Patterns, and Applications},
Saylor Foundation,
2011
\end{thebibliography}

\end{document}